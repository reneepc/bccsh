\documentclass{article}
\usepackage[utf8]{inputenc}
\usepackage[T1]{fontenc}
\usepackage{listings}
\usepackage{graphicx}
\usepackage{placeins}

\author{Renê Cardozo \\ 
        rene.cardozo@usp.br}

\title{BCCSH}

\date{}

\begin{document}
\maketitle

\section{Composição}

O executável bccsh é composto de dois módulos: parser e bccsh. Além disso é utilizada a biblioteca readline, a qual tem
seu caminho encontrado pela ferramenta pkgconfig.

\subsection{Parser}

O parser é responsável por receber a linha de caracteres retornada pela função readline da biblioteca readline e separar
cada palavra em um vetor de argumentos, os quais serão reconhecidos pelo bccsh posteriormente e determinarão qual
comando deverá ser executado.

Este parser foi adotado pensando em possibilitar a execução de qualquer binário pelo shell, bem como unificar o
processo de execução de comandos internos e outros programas do sistema.

\subsection{Bccsh}

O sheel propriamente dito é implementado no arquivo bccsh.c, o qual realiza a leitura de inputs pelo usuário utilizando
a função readline, sendo lida uma linha por vez. Após o armazenamento da linha em um buffer, caso a linha não seja
vazia, esta será armazenada no histórico através da função add\_history da biblioteca readline, podendo ser acessada,
por padrão, utilizando as setas do teclado.

Uma vez salvo no histórico, a linha será utilizada pelo parser para criar um vetor com as palavras digitadas. As quais
utilizarão a função cmd para determinar qual comando será executado. Podendo este ser:

\begin{itemize}
    \item mkdir <diretório>
    \item kill -<sinal> <PID>
    \item ln -s <arquivo> <link>
    \item Qualquer binário do sistema iniciado por "/" ou ".".
\end{itemize}

O parser não realiza o reconhecimento de strings dentro de aspas e separadas por espaços, uma vez que o reconhecimento
das palavras é feito através da sua divisão em cada espaço em branco encontrado pela função strtok da biblioteca
string.
\end{document}

