\documentclass{beamer}
\usepackage[utf8]{inputenc}
\usepackage[T1]{fontenc}
\usepackage{listings}
\usepackage{graphicx}
\usepackage{placeins}
\usetheme{Madrid}

\author{Renê Cardozo - 9797315 \\ 
        Verônica Stocco - 6828626 \\
        rene.cardozo@usp.br \\
        veronica.stocco@usp.br}

\title{EP1}
\institute{Instituto de Matemática e Estatística \\
Universidade de São Paulo}

\date{}

\begin{document}
\frame{\titlepage}

\begin{frame}

\frametitle{BCCSH}

O executável bccsh é composto de dois módulos: parser e bccsh. Além disso é utilizada a biblioteca readline, a qual tem
seu caminho encontrado pela ferramenta pkgconfig.

\end{frame}

\begin{frame}
\frametitle{Parser}

\begin{itemize}
\item O parser é responsável por receber a linha de caracteres retornada pela função readline da biblioteca readline e separar cada palavra em um vetor de argumentos, os quais serão reconhecidos pelo bccsh posteriormente e determinarão qual comando deverá ser executado.  

\item Este parser foi adotado pensando em possibilitar a execução de qualquer binário pelo shell, bem como unificar o processo de execução de comandos internos e outros programas do sistema.  


\item Há uma limitação de 10 strings para cada linha lida pelo parser.

\end{itemize}
\end{frame}

\begin{frame}

\begin{itemize}
\frametitle{Shell}
\item O shell propriamente dito é implementado no arquivo bccsh.c, o qual realiza a leitura de inputs pelo usuário utilizando
a função readline, sendo lida uma linha por vez. 

\item Após o armazenamento da linha em um buffer, caso a linha não seja
vazia, esta será armazenada no histórico através da função add\_history da biblioteca readline, podendo ser acessada,
por padrão, utilizando as setas do teclado.

\end{itemize}
\end{frame}

\begin{frame}
\frametitle{Comandos}

Uma vez salvo no histórico, a linha será utilizada pelo parser para criar um vetor com as palavras digitadas. As quais
utilizarão a função cmd para determinar qual comando será executado. Podendo este ser:

\begin{itemize}
    \item mkdir <diretório>
    \item kill -<sinal> <PID>
    \item ln -s <arquivo> <link>
    \item Qualquer binário do sistema iniciado por "/" ou ".", com no máximo 9 argumentos.
\end{itemize}
\end{frame}

\begin{frame}
\frametitle{Escalonador de Processos}
O simulador de escalonamento de processos é definido no arquivo ep1.c, no qual estão definidas as rotinas de
    escalonamento com os algoritmos First-Come, First-Served (FCFS) e Shortest Remaining Time Next (SRTN).

\begin{block}{Round-Robin}
O algoritmo Round-Robin não foi implementado por falta de tempo.
\end{block}

Neste arquivo é utilizada a função read\_file para ler os arquivos da pasta entrada, na qual são armazenados os arquivos
criados pelo gerador de processos, os quais são organizados em um array de structs onde é armazenado o nome, t0, dt e
    deadline de cada processo.
\end{frame}

\begin{frame}
\frametitle{Algoritmos de Escalonamento}
A partir do primeiro parâmetro fornecido para o executável ./ep1 temos a escolha do algoritmo que será utilizado na
    simulação, sendo 1 para FCFS e 2 para SRTN.

A implementação dos algoritmos foi feita da seguinte forma:
\begin{itemize}
    \item FCFS: Executa as threads por diretamente do array de processos montado na leitura, uma vez que supõe-se que os
        processos estejam ordenados por ordem de chegada.
    \item SRTN: Reordena o array de processos com base na soma dos atributos t0 e dt, assim, priorizam-se os processos
        com menor dt que chegam antes.
\end{itemize}

Os algoritmos armazenam a saída no arquivo especificado pelo terceiro parâmetro do executável ./ep1, no qual é escrito o
nome do processo o tempo atual do simulador e o tempo do simulador menos o tempo inicial do processo.

\end{frame}

\begin{frame}
\frametitle{Gerador de Processos}
A criação de processos para uso pelo simulador é feito pelo arquivo gerador.c dentro da pasta gerador\_de\_processos.
    Este programa é executado da forma "./gerador <número-de-processos [semente]", sendo a semente opcional e por
    padrão igual a 0.
\begin{itemize}
    \item t0 inicia-se com 1 e é incrementado em um inteiro aleatório entre 0 e 3.
    \item dt é gerado como um inteiro aleatório entre 2 e 5.
    \item a deadline do processo é gerada por t0 + dt + x, sendo x um número inteiro aleatório entre 0 e 15.
\end{itemize}
\end{frame}

\begin{frame}
\frametitle{Testes}
Os testes foram realizados a partir da geração de três arquivos de processos, com 5, 50 e 500 processos.

A execução de cada teste foi realizada para os dois algoritmos implementados e executadas trinta vezes para cada arquivo
de 5, 50 e 500 processos.

A execução usou como base:
\begin{itemize}
    \item ./bccsh
    \item ./ep1 <1 ou 2> entrada/<arquivo-de-entrada> saida/<arquivo-de-saida> d
\end{itemize}

\end{frame}

\begin{frame}
\frametitle{Máquinas Utilizadas}

Foram utilizadas duas máquinas diferentes, uma com 12 e outra com 8 CPUs, as quais estão descritas no diretório doc
    pelos arquivos doc/desc\_comp1 e doc/desc\_comp2 que registram as saídas do comando lscpu para cada uma das máquinas.


\begin{block}{Limitações}
A máquina 1, com 12 núcleos, sofreu aparente limitação pela utilização do Windows Subsystem for Linux, uma vez que
    apenas uma CPU parece ter sido utilizada, o que pode ser observado pela saída "0" no modo verboso do simulador.
\end{block}
\end{frame}
\end{document}

